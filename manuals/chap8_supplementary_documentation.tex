% !TEX TS-program = pdflatex
% !TEX encoding = UTF-8 Unicode

% This is a simple template for a LaTeX document using the "article" class.
% See "book", "report", "letter" for other types of document.

\documentclass[11pt]{article} % use larger type; default would be 10pt

\usepackage[utf8]{inputenc} % set input encoding (not needed with XeLaTeX)

\bibliographystyle{unsrt}

%%% Examples of Article customizations
% These packages are optional, depending whether you want the features they provide.
% See the LaTeX Companion or other references for full information.

%%% PAGE DIMENSIONS
\usepackage{geometry} % to change the page dimensions
\geometry{letterpaper, margin=1in} % or letterpaper (US) or a5paper or....
% \geometry{margin=2in} % for example, change the margins to 2 inches all round
% \geometry{landscape} % set up the page for landscape
%   read geometry.pdf for detailed page layout information

\usepackage{xcolor}
\usepackage{graphicx} % support the \includegraphics command and options
\usepackage{amsmath}
\usepackage{listings}
\lstset{language=[77]Fortran,
  basicstyle=\ttfamily\footnotesize,
  keywordstyle=\color{red},
  commentstyle=\color{blue}}
% \usepackage[parfill]{parskip} % Activate to begin paragraphs with an empty line rather than an indent

%%% PACKAGES
\usepackage{booktabs} % for much better looking tables
\usepackage{array} % for better arrays (eg matrices) in maths
\usepackage{paralist} % very flexible & customisable lists (eg. enumerate/itemize, etc.)
\usepackage{verbatim} % adds environment for commenting out blocks of text & for better verbatim
\usepackage{subfig} % make it possible to include more than one captioned figure/table in a single float
% These packages are all incorporated in the memoir class to one degree or another...

%\usepackage{hyperref}
\usepackage{url}

%%% HEADERS & FOOTERS
\usepackage{fancyhdr} % This should be set AFTER setting up the page geometry
\pagestyle{fancy} % options: empty , plain , fancy
\renewcommand{\headrulewidth}{0pt} % customise the layout...
\lhead{}\chead{}\rhead{}
\lfoot{}\cfoot{\thepage}\rfoot{}

%%% SECTION TITLE APPEARANCE
\usepackage{sectsty}
\allsectionsfont{\sffamily\mdseries\upshape} % (See the fntguide.pdf for font help)
% (This matches ConTeXt defaults)

%%% ToC (table of contents) APPEARANCE
\usepackage[nottoc,notlof,notlot]{tocbibind} % Put the bibliography in the ToC
\usepackage[titles,subfigure]{tocloft} % Alter the style of the Table of Contents
\renewcommand{\cftsecfont}{\rmfamily\mdseries\upshape}
\renewcommand{\cftsecpagefont}{\rmfamily\mdseries\upshape} % No bold!

%%% END Article customizations

%%% The "real" document content comes below...

\title{Brief Article}
\author{The Author}
%\date{} % Activate to display a given date or no date (if empty),
         % otherwise the current date is printed 

\begin{document}
%\maketitle

\section{Introduction}

This article provides additional guidance on Train Physical Data, supplementing section 8.2 of the SES Users' Manual.

\section{Acceleration Resistance of Rotating Parts}

The Acceleration Resistance of Rotating Parts is a factor which is used to increase the effective mass of the train to allow for the rotational inertia of the wheels, motors, axles, etc. 
The train effective mass consists of two parts: the empty car mass, which is modified by a factor to account for the acceleration resistance of rotating parts, and the mass of the passengers. 

The modified empty car mass is

\begin{equation}
m_\mathrm{v,mod} = m_\mathrm{v} \left(1 + f_\mathrm{acc}\right)
\label{eqn_effective_mass}
\end{equation}

\noindent where $f_\mathrm{acc}$ is a coefficient representing the effective mass of rotating parts relative to the car mass.
SES is coded in such a way that $f_\mathrm{acc}$ cannot be set directly. 
The corresponding user input is the Acceleration Resistance of Rotating Parts, $R_\mathrm{acc}$, with units lbf/(ton-mph/sec).
The two parameters are related by a unit conversion factor, $f_\mathrm{c}$,

\begin{align} \label{eqn_acc_resistance}
f_\mathrm{acc} &= \frac{R_\mathrm{acc}}{f_\mathrm{c}} \\
f_\mathrm{c} &= \frac{2000 \, \frac{\mathrm{lb} }{\mathrm{ton} } \times 5280 \frac{\mathrm{ft} }{\mathrm{mil} }}{32.174 \, \frac{\mathrm{ft}}{\mathrm s^2} \, \times 3600 \, \frac{\mathrm{s}}{\mathrm{hr}} } \\
 &= 91.2 \, \frac{\mathrm{lb}}{\mathrm{ton} } \frac{\mathrm {s \, hr}}{\mathrm{mil} } \nonumber
\end{align}

If the acceleration resistance of rotating parts is to be estimated as a fraction of the vehicle mass, then it can be converted to the appropriate SES input using equation \ref{eqn_acc_resistance}.
The default value of 8.8 lbs per ton/(mph/sec) corresponds to 9.6\% of the car mass. 

The implementation of equation \ref{eqn_effective_mass} in the SES source code is as follows.
The train acceleration is calculated in the {\tt TRAIN} subroutine (line 198), 

\begin{lstlisting}
C*****FIND ACCELERATION TRAIN IS CAPABLE OF                             
      DUDTV(NUMV)=(TEV(NUMV)*MOTORV(ITYP)-RSISTV(NUMV))/(WV(ITYP)*RRACC(
     1ITYP)+WPATV(NUMV)/GRACC) 
\end{lstlisting}

\noindent where the modified empty car mass (equation \ref{eqn_effective_mass}) appears as the term {\tt WV(ITYP)*RRACC(ITYP)}.
Note that the variable {\tt RRACC(ITYP)} is redefined when initially read by the {\tt GARAGE} subroutine (line 307):

\begin{lstlisting}
       RRACC(I) = ( 91.2 + RRACC(I) ) / ( 91.2 * GRACC )               
\end{lstlisting}

Equation \ref{eqn_effective_mass} can be recovered from the above source code as follows,

\begin{align*}
R_\mathrm{racc} &= \frac{91.2 + R_\mathrm{acc,input}}{91.2 g} = \frac{f_\mathrm{c} + R_\mathrm{acc,input}}{f_\mathrm{c} g} \\
m_{\mathrm{v,mod}} &= w_\mathrm{v} R_\mathrm{acc} = \frac{w_\mathrm{v} ( f_\mathrm{c} + R_\mathrm{acc,input} ) }{f_\mathrm{c} g} \\
 &= \frac{w_\mathrm{v}}{g} +  \frac{ w_\mathrm{v} R_\mathrm{acc,input}  }{f_\mathrm{c} g} \\
 &= m_\mathrm{v} +  \frac{ R_\mathrm{acc,input}  }{f_\mathrm{c}} m_\mathrm{v} \\
 &= m_\mathrm{v} \left(1 + f_\mathrm{acc}\right)
\end{align*}

\noindent where $w_\mathrm{v}$ is car weight, $g$ is gravitational acceleration and the subscript `input' indicates the original $R_\mathrm{acc}$ defined by the user.

\section{Rolling Resistance Coefficients}
The Train Rolling Resistance Coefficients are used to compute the mechanical friction created by
train movement, and suggested values for rubber-tired and steel wheels are provided on the instructions
for Input Form 9E. 

The mechanical rolling resistance is calculated in the subroutine {\tt GARAGE.FOR} (line 297) and {\tt TRAIN.FOR} (line 493) as

\begin{lstlisting}
      CORMV(I,2) = CORMV(I,2) * NCARV(I)
      RM=CORMV(ITYP,2)+(CORMV(ITYP,1)+CORMV(ITYP,3)*UV(NUMV))*WTRN
\end{lstlisting}   

\noindent which is equivalent to

\begin{align}
R_\mathrm{m} = C_1 w_\mathrm{t} + C_2 n_\mathrm{car} + C_3 w_\mathrm{t} u_\mathrm{v}
\end{align}

\noindent where $w_\mathrm{t}$ is the train weight (force in tons), $n_\mathrm{car}$ is the number of cars in a train and $u_\mathrm{v}$ is the train speed.

The default resistance coefficients for trains with steel wheels appear to be based on the original equation by Davis \cite{davis1926} (also see Szanto \cite{szanto2016}), who defined the specific resistance as

\begin{align}
r_\mathrm{Davis} = \frac{R_\mathrm{Davis}}{w_\mathrm{t}} =  1.3 + \frac{29}{w_\mathrm{ax}} + 0.045 u_\mathrm{v} + 0.0005 \frac{A u^2}{n_\mathrm{ax} w_\mathrm{ax}}
\end{align}

\noindent where $w_\mathrm{ax}$ is the axle load and $n_\mathrm{ax}$ is the number of axles per car.
Note that the axles are assumed to have journal bearings \cite{szanto2016}.
The default SES coefficients of $C_1 = 1.3$ lbf/ton, $C_2 = 116$ lbf and $C_3 = 0.045$ lbf/ton-mph can be recovered by assuming $n_\mathrm{ax}=4$ and noting that $n_\mathrm{ax} w_\mathrm{ax} = w_\mathrm{car}$ and $w_\mathrm{t}/w_\mathrm{car} = n_\mathrm{car}$.

\begin{align}
R_\mathrm{Davis} & = 1.3 w_\mathrm{t} + \frac{116 w_\mathrm{t}}{n_\mathrm{ax} w_\mathrm{ax}} + 0.045 w_\mathrm{t} u_\mathrm{v} + 0.0005 \frac{w_\mathrm{t} A u^2}{n_\mathrm{ax} w_\mathrm{ax}} \nonumber \\
 &= 1.3 w_\mathrm{t} + 116 n_\mathrm{car} + 0.045 w_\mathrm{t} u_\mathrm{v} + 0.0005 n_\mathrm{car} A u^2 
\end{align}

\noindent Note that the final term in the Davis equation accounts for aerodynamic resistance, and is not used in SES.

Szanto \cite{szanto2016} provides a useful discussion on the impact of roller bearings versus journal bearings as well as dynamic effects.

\bibliography{chap8bib}

\end{document}
